\documentclass{beamer}
\usepackage[utf8]{inputenc}
\mode<presentation>
{
  %\usetheme{default}
  %\usetheme{Boadilla}
  %\usetheme{Singapore}
  \usetheme{Warsaw}
  %\usetheme{Darmstadt}
  %\usetheme{Malmoe}
  %\usetheme{CambridgeUS}
  %\usefonttheme{serif}
  %\usecolortheme{wolverine}
  %\usecolortheme{beaver}
  %\setbeamercovered{transparent}
}

\usepackage{lmodern}
\usefonttheme{structurebold}
%\usefonttheme[onlymath]{serif}
\usefonttheme{professionalfonts}
%\setbeamerfont{vpiccolo}{size=\tiny}
%\setbeamercolor{background canvas}{bg=cyan} 
\setbeamertemplate{blocks}[rounded][shadow=true]
%\setbeamertemplate{footline}[page number]
\usepackage{color}
\definecolor{red}{rgb}{1,0,0} % color values Red, Green, Blue
\definecolor{green}{rgb}{0,1,0}
\definecolor{blue}{rgb}{0,0,1} % color values Red, Green, Blue
\definecolor{MyDarkBlue}{rgb}{0.1,0,0.55}

\usepackage[retainorgcmds]{IEEEtrantools}
\usepackage{graphicx}
\usepackage[italian]{babel}
\usepackage{amsmath}
\usepackage[T1]{fontenc}
\usepackage{amsfonts}
\usepackage{amsthm}
\usepackage{amssymb}
\usepackage{mathrsfs}
\def\N{\mathbb{N}}
\def\Z{\mathbb{Z}}
\def\Q{\mathbb{Q}}
\def\R{\mathbb{R}}
\def\C{\mathbb{C}}

%%%%%%%%%%%%%%%%%%%%%%%%%%%%%%%%%%%%%%
\begin{document}
\title[]{\bf Seminario di Determinazione Orbitale}

\author[Alessia Vanni]{\textcolor{MyDarkBlue}{\bf \large Alessia Vanni} }
    
%\titlegraphic{\includegraphics[width=20mm]{putto}}

\date{10/03/15}

\small
\begin{frame}[plain]
\titlepage
\end{frame}
%%%%%%%%%%%%
\begin{frame}[plain]
\tableofcontents
\end{frame}
%%%%%%%%%%%%%%%
\section{Introduzione al problema.}
%%%%%%%%%%%%%%%%%%%%%%%%%%%%%%%%%%%%%%%%%%%
\begin{frame}
Consideriamo il seguente sistema, composto dalla Terra e da un corpo che le orbita intorno. 
\vskip 0.5cm
%\begin{minipage}{.30\textwidth}
%\centering
%\includegraphics[ width=2cm, height=3cm ]{figura.png}
%\end{minipage}%
%\begin{minipage}{.70\textwidth}
\textbf{Dati iniziali}
\begin{itemize}
\item $n$ istanti $t_1, \dots, t_n$ .
\item $n$ vettori $\textbf{R}_k$ che rappresentano le posizioni dell'osservatore rispetto ad un sistema di riferimento fissato.
\item $n$ versori $ \hat{\mathbf{\rho}}_1, \dots, \hat{ \rho }_n$, osservati ai tempi $t_1, \dots, t_n$, che indicano la posizione del corpo rispetto alla Terra.
\end{itemize}


%\end{minipage}
\vskip 0.5cm



\end{frame}
%%%%%%%%%%%%%%%%%%%%%%%%%%%%%%%%%%%%%%%%%%%
\begin{frame}[plain]
Indicheremo inoltre con $\textbf{r}_k$ le posizioni del corpo rispetto al centro della Terra.
Lo scopo degli algoritmi presentati \`e quello di trovare le distanze topocentriche $\mathbf \rho_k$. 

I tre vettori soddisfano la relazione
 \begin{equation}
\textbf{r}_k=\textbf{R}_k+ \rho_k \hat{rho}_k, \quad k=1,\dots, n.
\end{equation}
Inoltre, dal momento che l'orbita del corpo giace su un piano, vale anche che \begin{equation}
\textbf{r}_k=c_k \textbf{r}_{k-1} + \textbf{r}_{k+1} \quad k=1,\dots, n
\end{equation}

\end{frame}
%%%%%%%%%%%%%%%%%%%%%%%%%%%%%%%%%%%%%%%%%%%
\begin{frame}[plain]
I coefficienti $c_k$ e $d_k$ possono essere ottenuti utilizzando le serie di Lagrange $f$ e $g$ per esprimere i vettori $\textbf{r}_{k-1}$ e $r_{k+1}$:
\begin{equation}
\begin{cases}
\textbf{r}_{k-1}=f_{k-1}\textbf{r}_k+g_{k-1}\textbf{v}_k \\
\textbf{r}_{k+1}=f_{k+1}\textbf{r}_k+g_{k+1}\textbf{v}_k
\end{cases}
\end{equation}
Questa equazione ci permette di eliminare il vettore velocit\`a $\textbf{v}_k$.
\end{frame}
%%%%%%%%%%%%%%%%%%%%%%%%%%%%%%%%%%%%%%%%%%%
\begin{frame}[plain]
Eliminando il vettore $\textbf{v}_k$ otteniamo la relazione che lega $\textbf{r}_k$ agli altri due:  \begin{equation}
\textbf{r}_k=\frac{g_{k+1}}{f_{k-1} g_{k+1}-f_{k+1} g_{k-1}} \textbf{r}_{k-1}- \frac{g_{k-1}}{f_{k-1} g_{k+1} - f_{k_1} g_{k-1}} \textbf{r}_{k+1}
\end{equation}
e quindi i coefficienti $c_k$ e $d_k$, $k=2,..., n-1$
\begin{equation}
c_k=\frac{g_{k+1}}{f_{k-1}g_{k+1}-f_{k+1}g_{k-1}} \qquad d_k=-\frac{g_{k-1}}{f_{k-1}g_{k+1}-f_{k+1}g_{k-1}}
\end{equation}
\end{frame}
%%%%%%%%%%%%%%%%%%%%%%%%%%%%%%%%%%%%%%%%%%%
\begin{frame}[plain]
I coefficienti $f_k$ e $g_k$ possono essere espansi in serie di Taylor fino al quarto ordine \begin{equation*}
\begin{cases}
f_{k-1} \approx 1-\frac{\mu}{2 r_k^3}\Delta t^2_k - \frac{\mu(\textbf{r}_k \cdot \textbf{v}_k)}{2 r_k^5} \Delta t^3_k + \frac{\mu}{24}[-2\frac{\mu}{r_k^6} + 3 \frac{v_k^2}{r_k^5} - 15 \frac{\textbf{r}_k \cdot \textbf{v}_k)}{r_k}]\Delta t ^4_k \\
f_{k+1}\approx 1-\frac{\mu}{2 r_k^3}\Delta t^2_k - \frac{\mu(\textbf{r}_k \cdot \textbf{v}_k)}{2 r_k^5} \Delta t^3_k + \frac{\mu}{24}[-2\frac{\mu}{r_k^6} + 3 \frac{v_k^2}{r_k^5} - 15 \frac{\textbf{r}_k \cdot \textbf{v}_k)}{r_k}]\Delta t ^4_k\\
g_{k-1} \approx 1-\frac{\mu}{2 r_k^3}\Delta t^2_k - \frac{\mu(\textbf{r}_k  \cdot \textbf{v}_k)}{2 r_k^5} \Delta t^3_k + \frac{\mu}{24}[-2\frac{\mu}{r_k^6} + 3 \frac{v_k^2}{r_k^5} - 15 \frac{\textbf{r}_k \cdot \textbf{v}_k)}{r_k}]\Delta t ^4_k\\
g_{k+1} \approx 1-\frac{\mu}{2 r_k^3}\Delta t^2_k - \frac{\mu(\textbf{r}_k \cdot \textbf{v}_k)}{2 r_k^5} \Delta t^3_k + \frac{\mu}{24}[-2\frac{\mu}{r_k^6} + 3 \frac{v_k^2}{r_k^5} - 15 \frac{\textbf{r}_k \cdot \textbf{v}_k)}{r_k}]\Delta t ^4_k
\end{cases}
\end{equation*}
e $\mu=398600.44 \quad km/s^2$ \`e il parametro gravitazionale della terra.
\end{frame}
%%%%%%%%%%%%%%%%%%%%%%%%%%%%%%%%%%%%%%%%%
\section{Definizione del primo algoritmo}


%%%%%%%%%%%%%%%%%%%%%%%%%%%%%%%%%%%%%%%%%%%
\begin{frame}[plain]
Iniziamo a definire il primo algoritmo per $n=3$ osservazioni. \\
Per piccoli intervalli di tempo sono sufficienti i primi de termini dello sviluppo di $f$ e $g$. \\
Inoltre se prendiamo $\Delta t$ costante \begin{equation}
c_k=d_k=\frac{1}{2} \left(1+ \frac{\mu}{2r_k^3} \Delta t^2 \right)
\end{equation} 
Possiamo adesso definire la funzione a $\psi: \R^3 \longrightarrow \R^3$ come \begin{equation}
\psi(\rho_1, \rho_2, \rho_3)=c_2(\textbf{R}_1+ \rho_1 \hat{\rho}_1)+d_2(\textbf{R}_3+ \rho_3 \hat{\rho}_3)-(\textbf{R}_2+ \rho_2 \hat{\rho}_2)
\end{equation}
\end{frame}

%%%%%%%%%%%%%%%%%%%%%%%%%%%%%%%%%%%%%%%%%%%
\begin{frame}[plain]
La soluzione $(\rho_1, \rho_2, \rho_3)$ cercata dovr\`a quindi soddisfare l'equazione vettoriale \begin{equation}
\psi(\rho_1^*, \rho_2^*, \rho_3^*)=\textbf{0}
\end{equation} 
e verr\`a cercata attraverso un metodo iterativo. \\
Scrivendo lo sviluppo di Taylor della funzione $\psi$ fino al secondo ordine, otteniamo che
\begin{equation}
\psi_{i+1} \approx \psi_i + J_i (\rho_{i+1} - \rho_i) + \frac{1}{2} \Delta \rho_i^T H_i \Delta \rho_i
\end{equation}
dove $\psi_{k} = \psi(\rho_{k})$ 
\end{frame}
%%%%%%%%%%%%%%%%%%%%%%%%%%%%%%%%%%%%%%%%%
\begin{frame}[plain]
E quindi il passo di iterazione \`e dato dalla formula \begin{equation}
\rho_i+1=\rho_i - J_i [\psi_i + \frac{1}{2} (J_i^{-1} \psi_i)^T H_i (J_i^{-1}\psi_i)]
\end{equation}
Se i tre vettori di posizione $(\textbf{R}_1,\textbf{R}_2, \textbf{R}_3)$ giacciono sullo stesso piano dei versori $(\hat{\rho}_1, \hat{\rho}_2,\hat{\rho}_3)$ allora la condizione di coplanarit\`a data dalla relazione (inserisci) collassa in due equazioni. In questo caso, avendo tre incognite $(\rho_1,\rho_2,\rho_3)$, \`e necessaria una quarta osservazione.  
\end{frame}
%%%%%%%%%%%%%%%%%%%%%%%%%%%%%%%%%%%%%%%%%%%
\begin{frame}[plain]
Questo metodo pu\`o essere facilmente esteso al caso in cui $n>3$. 
In questo caso le relazioni di coplanarit\`a sono date dal sistema \begin{equation}
\begin{cases}
\textbf{r}_2=c_2 \textbf{r}_1 + d_2 \textbf{r}_3 \\
\textbf{r}_3=c_3 \textbf{r}_3 + d_3 \textbf{r}_4 \\
\quad \vdots \\
\textbf{r}_{n-1}=c_{n-1} \textbf{r}_{n-2} + d_{n-2} \textbf{r}_n
\end{cases}
\end{equation}
\end{frame}

%%%%%%%%%%%%%%%%%%%%%%%%%%%%%%%%%%%%%%%%%%%%%%%%%
\begin{frame}[plain]
L'equazione che descrive i residui diventa quindi 
\begin{equation}
\begin{cases}
\psi_1=c_2(\textbf{R}_1+\rho_1 \hat{\rho}_1) + d_2(\textbf{R}_3+\rho_3 \hat{\rho}_3)-(\textbf{R}_2+\rho_2 \hat{\rho}_2) \\
\vdots \\
\psi_{n-2}= c_{n-1} (\textbf{R}_{n-2}+\rho_{n-2} \hat{\rho}_{n-2}) + d_{n-1}(\textbf{R}_n+\rho_n \hat{rho}_n) \\ 
  \qquad - (\textbf{R}_{n-1}+\rho_{n-1} \hat{\rho}_{n-1})
\end{cases}
\end{equation}
Analogamente al caso precedente, gli zeri di questa equazione vengono calcolati con il metodo di Newton, con la differenza che la matrice Jacobiana ha dimensione $3(n-2) \times n$ e non \`e necessariamente quadrata. 
\end{frame}

%%%%%%%%%%%%%%%%%%%%%%%%%%%%%%%%%%%%%%%%%%%%%%%%%
\begin{frame}[plain]
Tralasciando le derivate di ordine superiore, l'iterazione diventa \begin{equation}
\rho_{i+1}=\rho_i - [(J_i^T J_i)^{-1}J_i]\psi
\end{equation}
Il prossimo metodo che andremo a presentare non richiede la costruzione di alcuna derivata.
\end{frame}


%%%%%%%%%%%%%%%%%%%%%%%%%%%%%%%%%%%%%%%%%%%%%
\section{Soluzione ai minimi quadrati}
%%%%%%%%%%%%%%%%%%%%%%%%%%%%%%%%%%%%%%%%%%%%%%%%%
\begin{frame}
\`E possibile riscrivere le condizioni di coplanarit\`a \begin{equation}
c_k \rho_{k-1}\hat{\rho}_{k-1}-\rho_{k}\hat{\rho}_{k} +d_k \rho_{k+1}\hat{\rho}_{k+1}=\textbf{R}_k-(c_k\textbf{R}_{k-1}+d_k \textbf{R}_{k+1})
\end{equation}
nella forma compatta $M \rho =\xi$:
\begin{equation*}
\begin{bmatrix}
c_2 \hat{\rho}_1 & -\rho_2 & d_2 \hat{\rho}_3 & \mathbf{0} & \mathbf{0}  & \dots & \mathbf{0} \\
\mathbf{0} & c_3 \hat{\rho}_2 & -\rho_3 & d_3 \hat{\rho}_4 & \mathbf{0} & \dots &  \mathbf{0} \\
\mathbf{0} & \mathbf{0}  & c_4 \hat{\rho}_3 & -\rho_4 & d_4 \hat{\rho}_5 & \dots \mathbf{0} \\
\vdots & \vdots & \vdots & \vdots & \vdots & \ddots & \vdots \\
\mathbf{0}  & \mathbf{0}  & \mathbf{0}  & \mathbf{0}  & \mathbf{0}  & \dots & d_{n-1} \hat{\rho}_{n}
\end{bmatrix}
\begin{bmatrix}
\rho_1 \\
\rho_2 \\
\vdots \\
\rho_n
\end{bmatrix}=\begin{bmatrix}
\xi_1\\
\xi_2 \\
\vdots \\
\xi_n
\end{bmatrix}
\end{equation*}
\end{frame}
%%%%%%%%%%%%%%%%%%%%%%%%%%%%%
\begin{frame}[plain]
La soluzione ai minimi quadrati \`e data dalla formula \begin{equation}
 \rho= (M^T M)^{-1} M^T \xi
\end{equation}
Per trovare il punto fisso della successione si proceder\`a con l'iterazione di punto fisso \begin{equation}
 \rho^{(i)} = -(M^{(i)T} M^{(i)})^{-1} M^{(i)T} \xi^{(i)}
\end{equation}
avente $\rho^{(0)}=\textbf{0}$ come pu
\end{frame}
%%%%%%%%%%%%%%%%%%%%%%%%%%%%%
\section{Esempi e risultati}
%%%%%%%%%%%%%%%%%%%%%%%%%%%
\begin{frame}
Per controllare l'esattezza e la stabilit\`a del metodo sono stati analizzati quattro diversi casi. \\
 \begin{itemize}
\item Satellite equatoriale
\item Satellite con orbita inclinata di 45 gradi
\item Satellite con orbita inclinata di 90
\item Asteroide
\end{itemize}
\end{frame}
%%%%%%%%%%%%%%%%%%%%%%%%%%%%%%%%%%

\begin{frame}[plain]
In ognuno di essi il sito d'osservazione \`e posto a latitudine e longitudine nulle. I test sono stati eseguiti seguendo determinati passi: \begin{itemize}
\item[a)] Sono stati selezionati gli elementi orbitali di un oggetto di tipo LEO. La selezione deve esssere cosistente con la posizione dell'osservatore;
\item[b)] Dai dati selezionati vengono ottenuti i vettori posizione e velocit\`a iniziali;
\item[c)] Velocit\`a e posizione iniziali vengono propagate nel tempo per ottenere le osservazioni seguenti;
\item[d)] Si calcolano i versori $\hat{\rho}_k$;
\end{itemize}
\end{frame}
%%%%%%%%%%%%%%%%%%%%%%%%%%%%%%%%%
\begin{frame}[plain]
Per simulare delle osservazioni reali l'ascensione retta e la declinazione della direzione del corpo sono state "sporcate" con del rumore gaussiano.
L'accuratezza dei risultati ottenuti viene valutata in percentuale, confrontando i dati reali con quelli ottenuti:
\begin{equation}
err=100*\frac{|r_{true}-r_{est}|}{|r_{true}|}
\end{equation}
\end{frame}
%%%%%%%%%%%%%%%%%%%



\begin{frame}[plain]
Faremo adesso vedere che l'accuratezza dei risultati dipende principalmente da tre fattori: \begin{itemize}
\item L'intervallo di tempo che separa le osservazioni;
\item Gli errori di misurazione;
\item Gli elementi orbitali;
\end{itemize}
I test sono stati eseguiti sul secondo caso. \\
Per studiare l'effetto del rumore la varianza $\sigma$ \`e stata fatta variare da $0$ a $5''$.
\end{frame}
%%%%%%%%%%%%%%%%%%%


%%%%%%%%%%%%%%%%%%%%%%%%%%%%%%%%%%%%%%%
\begin{frame}[plain]
Per verificare l'influenza dell'intervallo di tempo, la varianza del rumore \`e stata fissata a $\sigma=5''$ e l'intervallo di tempo \`e stato fatto variare da $20$ a $120s$. L'incremento dell'intervallo di tempo aumenti l'accuratezza di entrambi i metodi fino ai $60s$. Ma per intervalli pi\`u lunghi $J_n$ risente dell'approssimazione dei coefficienti di Lagrange $f$ e $g$.
\begin{figure}
\centering
\includegraphics[width=0.5\textwidth]{errore2.png}
\end{figure}
\end{frame}
%%%%%%%%%

\begin{frame}[plain]
\begin{figure}
\centering
\includegraphics[width=0.5\textwidth]{errore_1.png}
\end{figure}

La figura mostra che per dati molto accurati la soluzione trovata con l'algoritmo $L_n$ \`e molto precisa, mentre quella trovata con $J_n$ non lo \`e. 

Al contrario, quando la varianza \`e superiore a $2''$ i due metodi hanno lo stesso livello di accuratezza.   
Questo \`e dovuto al troncamento al secondo ordine dei coefficienti $c_k$ e $d_k$
\end{frame}
%%%%%%%%%%%%%%%%%%%%%%%%%%%%%%%%%%%%

%\begin{frame}
%Infine l'ultima figura mostra come cambia la performance di $J_n$ al variare delle %condizioni iniziali. \\
%In questo caso l'intervallo di tempo rimane fisso a $50 s$. 
%\begin{figure}
%\centering
%\includegraphics[width=0.5\textwidth]{errore3.png}
%\end{figure}
%\end{frame}
%%%%%%%%%%%%%%%%%%%%%%%%%%%%%%%%
\section{Esempi e risultati sperimentali}
\begin{frame}
Sono stati condotti ulteriori test in MatLab, esguiti nuovamente sul secondo caso, con tre osservazioni. \begin{itemize}
\item Posizione dell'osservatore: $\textbf{R}_1=[6 378.137, 0, 0]^T$;
\item Condizioni iniziali: \begin{eqnarray*}
\textbf{r}_1 &=&[6882.26672, -514.02760,1284.74917 ]^T km \\
\textbf{v}_1 &= &[-0.5787, 5.7434, 5.3979]^T km/s;
\end{eqnarray*}
\end{itemize}
I coefficienti $c_k$ e $d_k$ sono nella versione approssimata \begin{equation}
c_k=d_k=\frac{1}{2} \left(1+ \frac{\mu}{2r_k^3} \Delta t^2 \right)
\end{equation}
sia in $J_3$ che in $L_3$
\end{frame}

%%%%%%%%%%%%%%%%%%%%%%%%%%%%%%%
\begin{frame}[plain]
\textbf{Ricerca dei dati iniziali}\begin{itemize}
\item Si propagano le condizioni iniziali $\textbf{r}_0$ e $\textbf{v}_0$ per ottenere le ossevazioni agli istanti $t_2s$e $t_3$;
\item Si calcolano i vettori $\textbf{R}_2$ e $\textbf{R}_3$, di coordinate \begin{equation}
\textbf{R}_2=\begin{pmatrix}
R_1*cos(\omega t_2) \\
R_1*sin(\omega t_2) \\
0
\end{pmatrix}
, \qquad \textbf{R}_3=\begin{pmatrix}
R_1*cos(\omega t_3) \\
R_1*sin(\omega t_3) \\
0
\end{pmatrix}
\end{equation}  
dove $w$ \`e l velocit\`a angolare della Terra; 
\item Si trovano le posizioni relative \begin{equation*}
\mathbf{\rho}_i=\textbf{r}_i-\textbf{R}_i, \quad i=1,2,3.
\end{equation*}
\end{itemize}
\end{frame}
%%%%%%%%%%%%%%%%%%%%%%%%%%%%%
\begin{frame}
\textbf{Modifica dei dati iniziali:}
\begin{itemize}
\item Si calcolano i moduli $(\rho_1, \rho_2, \rho_3)$ versori $\hat{\rho}_1,\hat{\rho}_2,\hat{\rho}_3$;
\item Si portano i versori in coordinate sferiche $(\alpha_i, \delta_i, 1), \quad i=1,2,3$;
\item Si modificano le prime due coordinate $(\alpha_i, \delta_i), \quad i=1,2,3$ e si torna in coordinate cartesiane.
\end{itemize}
\end{frame}
%%%%%%%%%%%%%%%%%%%%%%%%%%%%%%
\begin{frame}[plain]

\begin{figure} 
\centering
\includegraphics[width=0.5\linewidth]{error_vs_noise.jpg} 
\end{figure}
A questo punto si lanciano gli algoritmi $J_3$ e $L_3$ facendo variare il rumore da $0$ a $5''$. \\
In questo test il vettore di partenza $\rho^{(0)}$ di $J_3$ \`e costituito dai moduli veri.
\end{frame}
%%%%%%%%%%%%%%%%%%%%%%%%%%%%%%%%%%%%%
\begin{frame}[plain]
\begin{figure} 
\centering
\includegraphics[width=0.5\linewidth]{error_vs_time.jpg} 
\end{figure}
Se invece si vuol far variare l'intervallo di tempo $\Delta t$ tra un'osservazione e l'altra, si devono calcolare ogni volta i dati da inserire negli algoritmi.
\end{frame}

%%%%%%%%%%%%%%%%%%%%%%%%%%%%%%%
\begin{frame}
\begin{figure}
\includegraphics[width=0.5\linewidth]{occorrenze_positive_jn.jpg} 
\includegraphics[width=0.5\linewidth]{occorrenze_positive_dc.jpg}
\end{figure}
Occorrenze degli errori relativi, eseguito su 2000 prove.
\end{frame}
%%%%%%%%%%%%%%%%%%%%%%%%%%%%%%%%%
\begin{frame}[plain]
\begin{figure} 
\centering
\includegraphics[width=0.5\linewidth]{occorrenze_jn.jpg} 
\includegraphics[width=0.5\linewidth]{occorrenze_dc.jpg} \\
\end{figure}
Errore relativo: \begin{equation}
err=100*\frac{|r_{est}|-|r_{true}|}{|r_{true}|}
\end{equation}
\end{frame}
%%%%%%%%%%%%%%%%%%%%%%%%%%%%%%%%%%%%%%%%%%%%%%%%%
\begin{frame}[plain]
\begin{figure}
\centering
\includegraphics[width=0.7\textwidth]{guess_region.jpg}
\end{figure}
Regione di convergenza di $J_3$.
\end{frame}

\end{document}
